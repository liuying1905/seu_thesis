\chapter{可能用得上的模板}
\label{introduction}

\section{引用}

深度强化学习\cite{Mnih2013PlayingAW,Ye2020MasteringCC}中,深度学习用于复杂环境的状态表征,有效抽取图像、仿真等高维连续状态空间的特征;强化学习通过与环境交互获得经验存入经验池,采样经验更新网络,学习最大化期望回报的目标策略\upcite{Silver2016MasteringTG,Li2017DeepRL}。

引用文章使用 cite,引用文章内容使用 upcite,多个引用使用英文逗号间隔。

\section{图片}

\subsection{单张图片}

\begin{figure*}
	\centering
	\includegraphics[width=0.7\textwidth]{img/ohana.png}
	\caption{\label{fig:one_fig}单张图片}
\end{figure*}

单张图片如图\ref{fig:one_fig}所示。

\subsection{多张图片}

\begin{figure*}[htbp]
	\centering
	
	\subfigure{
		\begin{minipage}[t]{0.5\linewidth}
			\centering
			\includegraphics[width=2.8in]{img/ohana.png}\\
			\vspace{0.02cm}
			\includegraphics[width=2.8in]{img/ohana.png}\\
			\vspace{0.02cm}
		\end{minipage}%
	}%
	\subfigure{
		\begin{minipage}[t]{0.5\linewidth}
			\centering
			\includegraphics[width=2.8in]{img/ohana.png}\\
			\includegraphics[width=2.8in]{img/ohana.png}\\
		\end{minipage}%
	}%
	
	\centering
	\caption{多张图片}
	\label{fig:multi_fig}
	
\end{figure*}

\newpage
\section{表格}

\subsection{表格一}
表格一如表\ref{tab:tab_1}所示,表格中可以引用参考文献方便阅读。
\begin{table}[htbp]
	\caption{表格一}
	\label{tab:tab_1}
	\renewcommand\arraystretch{1.2}
	%\Large
	\begin{tabular}{L{2.5cm} L{3.2cm} L{4cm} L{4.5cm}}
		%		{>{\normalsize}m{0.2\textwidth} <{\centering}m{0.2\textwidth}<{\centering}m{0.2\textwidth} <{\centering}m{0.3\textwidth}}
		\hline
		方法 	& 经验组成		& 经验池		& 保留优先级			\\
		\hline
		方法一\cite{Mnih2013PlayingAW} & 转移	& 先进先出	 & 时间顺序		\\	
		方法二\cite{Karimpanal2018ExperienceRU} 	& 转移序列	& 先进先出	 & 时间顺序		\\	
		\hline
	\end{tabular}
\end{table}

\subsection{表格二}
表格二如表\ref{tab:tab_2}所示。
\begin{table*}[htbp]
	\centering
	\caption{\label{tab:tab_2}表格二}
	\begin{tabular}
		{C{3cm} C{1.2cm} C{1.5cm}  C{1.2cm} C{1.5cm}  C{1.2cm} C{1.5cm}}
		\hline
		\multirow{2}{*}{数据集}   &   \multicolumn{2}{c}{方法一} &   \multicolumn{2}{c}{方法二} &   \multicolumn{2}{c}{方法三}  \\
		%		\cline{2-9}
		&	指标1 & 指标2 &	指标1 & 指标2&	指标1 & 指标2	\\
		\hline
		数据集1	&	0.975	&	0.53	&	0.983	&	0.56	&	$\bm{1.000}$	&	$\bm{0.32}$	\\
		数据集2	&	0.975	&	0.53	&	0.983	&	0.56	&	$\bm{1.000}$	&	$\bm{0.32}$	\\
		\hline\hline
		平均值	& 0.989	& 0.31	& 0.988	& 0.29	& $\bm{1.000}$	& $\bm{0.17}$ \\
		\hline
	\end{tabular}
\end{table*}

\section{公式}

需要等号对齐的公式写法。

\begin{equation}
\begin{split}
\nabla_{\theta}J(\pi_{\theta})
& = \int_{\mathcal{S}}\rho^{\pi}(s)\int_{\mathcal{A}} \nabla_{\theta}\pi_{\theta}(a|s)Q^{\pi}(s,a)dads \\
& = \mathbb{E}_{s\sim\rho^{\pi},a\sim\pi_{\theta}}[\nabla
_{\theta}\log\pi_{\theta}(a|s)Q^{\pi}(s,a)]
\end{split}
\end{equation}

\section{算法}

算法如算法\ref{algorithm:alg_1}所示。

\begin{algorithm}[htbp]
	\caption{算法}
	\label{algorithm:alg_1}
	\begin{algorithmic}[1]
		\Require 输入数据
		\Ensure 输出结果
		\State 步骤
		\For {时间步t = $0, \dots, T$}
		\State 循环
		\If {判断条件}
		\State 步骤
		\EndIf
		\State 步骤
		\EndFor
	\end{algorithmic}
\end{algorithm}